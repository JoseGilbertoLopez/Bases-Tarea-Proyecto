\documentclass{article}

\usepackage[T1]{fontenc}
\usepackage[utf8]{inputenc}
\usepackage[spanish]{babel}

\title{(Tarea 5:) Dependencias y Normalización}
\author{Naranjo Robledo Carlos \\ Lopéz García José Gilberto}

\begin{document}
  \maketitle{Tarea 5}
\begin{enumerate}
  \item[(1)] Preguntas de repaso.
  \begin{enumerate}
    \item[(1)] ¿Qué es una dependencia funcional y comó se define?
    Se trata de una relación unididreccional entre 2 atributos de tal forma que
    en un moment dado, para cada valor único de $A$, sólo un valor de $B$ se
    asocia con él através de la relación.
    \item[(2)] ¿Para qué sirve el concepto de dependencia en la normalización?
    Ayudan a especificar formalmente cuando un diseño es correcto.
    \item[(3)] Sea $A$ la llave de $R(A,B,C)$. indica todas las dependencias
    funcionales que implica $A$.
    $$ A += \{A,B,C\} $$
    \item[(4)] ¿Qué es una forma normal? ¿Cuál es el objetivo de normalizar un
    modelo de datos?

    \item[(5)] En qué casos es preferible lograr 3NF en vez de BCNF
  \end{enumerate}
\item[(4)] Para cada una de las siguientes relaciones con su respectivo conjunto de dependencias funcionales:\\
$a. R(A,B,C,D,E,F) con F = \{B\rightarrow D, B \rightarrow E, D \rightarrow F, AB \rightarrow C\}\\$
$b. R(A,B,C,D,E) con F = \{A \rightarrow BC, B \rightarrow D, CD \rightarrow E, E \rightarrow A\}\\$
$Indica todas las violaciones a la 3NF$\\
$Normaliza de acuerdo a la 3N$\\
$A.  R(A,B,C,D,E,F) con F = \{B \rightarrow D, B \rightarrow E, D \rightarrow F, AB \rightarrow C\}$\\
$F=\{B \rightarrow DE, D \rightarrow F, AB \rightarrow C\}$\\
$Superfluo Izquierdos:\\
Es evidente en este caso que no existe variables superfluos del lado izquierdo, solo la dependencia funcional AB \rightarrow C$  cumple tener mas de dos variables del lado izquierdo y B solo llega a D y E entonces no hay forma posible de llegar a C sin A.\\
$Superfluo Derecho:$\\
$B \rightarrow DE$\\
$¿D superfluo?$\\
$F'\{B \rightarrow E, D \rightarrow F, AB \rightarrow C\}$\\
$\{B\}+=\{B,E\} \quad D no es superfluo.$\\
$¿E es superfluo?$\\
$F'\{B \rightarrow D, D \rightarrow F, AB \rightarrow C\}$\\
$\{B\}+=\{B,D,F\} \quad E no es superfluo$\\
$F tenia desde el principio el minimo conjunto de dependencias funcionales$\\
$B. R(A,B,C,D,E) con F = \{A \rightarrow BC, B \rightarrow D, CD \rightarrow E, E \rightarrow A\}$\\
$Superfluo Izquierdo:$\\
$CD \rightarrow E$ \\
$C superfluo$\\
$D \rightarrow E$\\
$\{D\}+=\{D,E,A,B,C\} \quad C es superfluo$\\
$F=\{A \rightarrow BC, B\rightarrow D, D\rightarrow E, E \rightarrow A\}$ \\
$Superfluo Derecho:$\\
$A\rightarrow BC$\\
$¿B superfluo?$\\
$F'=\{A \rightarrow C, B \rightarrow D, D\rightarrow E, E\rightarrow A\}$\\
$\{A\}+=\{A,C\} \quad B no es superfluo$\\
$C superfluo$\\
$F'=\{A \rightarrow B, B \rightarrow D, D\rightarrow E, E \rightarrow A\}$\\
$\{A\}+=\{A,B,D,E\} \quad C no es superfluo$\\
$F ya tiene el minimo conjunto de dependencias funcionales al terminar de ver los superfluos izquierdos$\\
\item[(5)] Sea el esquema:\\
$R(A,B,C,D,E,F) con F=\{BD \rightarrow E, CD \rightarrow A, E \rightarrow C, B \rightarrow D\}$\\
¿Qué puedes decir de $\{A\}+ y \{F\}+$?

Calcula \{B\}+, ¿qué puedes decir de esta cerradura?\\
Obtén todas las llaves candidatas.\\
¿R cumple con BCNF? ¿Cumple con 3NF? (en caso contrario normaliza)\\
Se ha decidido dividir R en las siguientes relaciones S(A,B,C,D,F) y T(C,E), ¿se puede recuperar la
información de R?\\
$\{A\}+=\{A\}$\\
$\{F\}+=\{\}$\\
$\{B\}+=\{B,D,E,C,A\}$\\
Llave candidata: B\\
3ra Forma Normal\\
Superfluo Izquierdo\\
$BD \rightarrow E$\\
¿$B superfluo$?\\
$D\rightarrow E$\\
$\{D\}+=\{D,E,C,A\} \quad B no es superfluo$\\
$¿D superfluo?$\\
$B\rightarrow E$\\
$\{B\}+=\{B,D,E,C,A\} \quad D es superfluo$\\
$F=\{B \rightarrow E, CD \rightarrow A, E \rightarrow C, B\rightarrow D\}$\\
$F=\{B \rightarrow DE, CD \rightarrow A, E \rightarrow C\}$\\
$¿C superfluo?$\\
$D\rightarrow A$\\
$\{D\}+=\{D,A\} C no es superfluo$\\
$¿D superfluo?$\\
$C \rightarrow A$\\
$\{C\}+=\{C,A\} \quad D no es superfluo$\\
$Superfluo Derecho:$\\
$B \rightarrow DE$\\
$¿D superfluo?$\\
$F'=\{B \rightarrow E, CD \rightarrow A, E\rightarrow C\}$\\
$\{B\}+=\{B,E,C\} \quad D no es superfluo$\\
$¿E superfluo?$\\
$F'=\{B \rightarrow D, CD \rightarrow A, E \rightarrow C\}$\\
$\{B\}+=\{B,D\} \quad E no es superfluo$\\
$F ya tenia el conjunto de dependencias funcionales al terminar los superfluo izquierdo$\\
$La llave de R es: B$\\
$Dividimos en particiones$\\
S(B,D,E) $con B \rightarrow DE$\\
T(C,D,A) $con CD \rightarrow A$\\
U(E,C) $con E \rightarrow C$\\
La llave de R esta contenido en S entonces ya esta normalizado\\
$4ta Forma Normal$\\
$Llave R: B$\\
$B\rightarrow DE$ \\$Esta dependencia cumpla con la regla de normalizacion$\\
$las demas dependencias son violaciones a la 4ta forma normal$\\
S(E,C) con $E \rightarrow C$, $una llave para S es E normalizado$\\
T(E,C,D,A)$ Llave de T = ECD es violacion$\\
Sea:\\
U(C,D,A) Llave de U = CD y la dependencia es $CD \rightarrow A$ normalizado\\
V(C) $C\rightarrow C \quad es trivial ya esta normalizado$\\
\\
\item[(6)]Para cada uno de los esquemas, con su respectivo conjunto de dependencias multivaluadas,
resuelve los siguientes puntos:\\
a. R(A,B,C,D) con DMV = $ \{ AB \rightarrow \rightarrow C, B \rightarrow D \} $\\
b. R(A,B,C,D,E) con DMV = $\{ A \rightarrow \rightarrow B, AB \rightarrow C, A \rightarrow D, AB \rightarrow E\}$\\
$Encuentra todas las violaciones a la 4NF$\\
$Normaliza de acuerdo a la 4NF$\\
a. Llave para R: AB
$AB\rightarrow\rightarrow C$\\
$B\rightarrow D \quad violacion$\\
S(B,D) con $B \rightarrow D \quad Llave B para S normalizado$\\
T(B,A,C) con $AB\rightarrow\rightarrow C$\\
Con $DMV=\{AB\rightarrow\rightarrow C\} $ La llave es AB que es la llave de R esta normalizado\\
\\
$b. Llave para R: AB$\\
$DMV= \{A \rightarrow\rightarrow B, AB\rightarrow CE,  A\rightarrow D \}$\\
$AB \rightarrow CE \quad Ya esta normalizado$\\
$A \rightarrow D \quad violacion$\\
S(A,D)$ con A\rightarrow D$ La llave de S es A normalizado\\
T(A,C,E,B) con $DMV\{A\rightarrow\rightarrow B\} \quad La llave es AB$\\
U(A,B,C,E) con $Ab \rightarrow CE Llave AB que es la llave de R normalizado$\\
V(A,B) con $A\rightarrow\rightarrow B Trivial Normalizado$\\
\end{enumerate}

\end{document}
